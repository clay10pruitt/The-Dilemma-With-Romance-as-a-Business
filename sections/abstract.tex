% \begin{abstract}
%     Tinder is likely to experience a decline in market growth in the near future and must find a way to maintain profitability without sacrificing user satisfaction. We apply the stable marriage problem to set theory, graph theory, and ultimately network science to determine how Tinder may match users. Creating a model using a preference network and the stable marriage problem allows us to find features of Tinder that lend to profitability and user satisfaction. We determine that premium features are optimal as they deter users from leaving Tinder while increasing the revenue the service earns.
% \end{abstract}
\begin{abstract}
    It is paradoxical for an optimal online dating service to exist since a service which pairs all of its users will have no customers. As such, Tinder's algorithms must balance the user's desire to be paired in a relationship with Tinder's need for enough users to run a successful business. However, these algorithms are proprietary and confidential, requiring an outside analysis in order to determine how they function. We start by observing applications of the stable marriage problem in set theory, graph theory, and network science to determine how an online dating service can optimally match users. We then introduce the concept of reciprocal recommender systems to strengthen the connection between the stable marriage problem and Tinder. These insights are used to create a model for the optimal online dating service to show that Tinder is capable of performing optimally, but must de-optimize itself to prevent pairing all users in relationships. It is determined that premium features are strong de-optimization methods as they keep users within Tinder's user base while also increasing the service's revenue.
\end{abstract}