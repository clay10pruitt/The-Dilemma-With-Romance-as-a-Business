An \textit{online dating service} is a platform designed to help a person, or \textit{user}, find their ideal partner. Users will provide the service with personal information, such as age, gender, or sexual preference \citep{Courtois2018}, that will constitute the user's \textit{profile}. Profiles are then input into a \textit{recommender algorithm} \citep{Andrews2015} to determine which users should be recommended to one another based off the value, or \textit{utility}, of the recommendation. If two users who have been recommended to each other express reciprocal interest, a channel of communication is opened to allow further exploration of interest \citep{Courtois2018}. These services have proven to be quite successful, spurring a rapid market growth. However, analysts predict that growth will start to decrease significantly in the near future \citep{Leskin2019}. 

One possible explanation is that users who form relationships no longer have use of an online dating service. With this in mind, it could be stated that online dating services are self-destructive. By fulfilling their advertised purpose of forming relationships, they will diminish the size of their target audience with every success. Without enough users, the service will be unable to operate as a successful business. Thus, there appears to be a paradox that prevents an \textit{optimal online dating service} from existing. Online dating services must find ways to \textit{de-optimize} if they wish to continue as a business. However, users will also cease to use a service that performs poorly. There must then be a balance between service profitability and user satisfaction. This balance appears to be highly influential in algorithms utilized by the online dating service Tinder.

Tinder has been described as a mobile dating app that enables the discovery of nearby singles \citep{Seidman2017}. As of 2020, the service is utilized by fifty-seven million users globally across one-hundred and ninety countries \citep{Iqbal2020}. An initial set of features are provided free of charge, but additional \textit{premium features} are offered at a cost \citep{Tinder2020}. 

A simplistic summary of Tinder's business model can be made with this information. Introduce users to a limited selection of the service's features and then tempt them to pay for additional features that claim to improve the user's chances of finding success \citep{Courtois2018}. This model appears to be quite successful, as the service generated over a billion dollars in revenue in 2019. In fact, we will find that this model is Tinder's strongest de-optimizer as premium features encourage users to be satisfied with recommendations that hold low utility. %Premium features may explain this, as the improved chances for success are determined entirely by Tinder.

In the first section of this paper we cover essential background knowledge in the stable marriage problem through set theory, graph theory, and network science. The results of these analyses are used to construct a model for an optimized online dating service in the second section. In the third section we compare the model with Tinder's operations to determine how premium features and other minor methods encourage de-optimization. We conclude by summarizing our findings and suggesting areas of future work.