\subsection{Tinder}

Tinder describes itself as ``the world's most popular app for meeting new people" \citep{TinderWhatIs}, as opposed to just an online dating service. Similar to most online dating services, users are able to create profiles where they can specify their personal preferences, upload photos, and write small biographies about themselves \citep{Courtois2018}. Users can then either \textit{like} or \textit{dislike} other users recommended by Tinder \citep{Courtois2018}. Liking and disliking is also referred to as \textit{swiping right} or \textit{swiping left} respectively, or as just \textit{swiping} when referring to the general act of giving an opinion. This naming scheme is in reference to opinions being delivered by swiping a graphical representation of a profile to the left or right side of the user's screen \citep{TinderWhatIs}. A \textit{match} is formed when two users like each other. A private channel of communication is then opened between them \citep{Courtois2018}.

Users are not informed when they receive a regular like \citep{TinderWhatIs}. However, Tinder also offers one free \textit{super like} to users per day \citep{TinderSuperLike}. In contrast to regular likes, users are informed from whom a super like came \citep{TinderSuperLike}. Additional super likes can be purchased \citep{TinderSuperLike}; this and other premium features will be explored in later sections.

While the service advertises itself as a means for meeting new people, it also claims to host a large community of singles \citep{TinderWhatIs}. It has also been found that the service is typically used for finding both romantic and sexual partners as well as to form relationships \citep{Gatter2016}. Thus, it is not unreasonable that Tinder can be thought of as an online dating service (as opposed to a social network such as Facebook). The question then is how can Tinder fulfill its obligations as an online dating service while remaining profitable? 