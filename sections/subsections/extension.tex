\subsection{Extension of the Model to Tinder}

A simple model can be constructed for Tinder using the arguments made previously. While a full mathematical model is outside the scope of this paper, the framework can be constructed informally. Let $U(t)$ be the size of Tinder's user base at time $t$ and let $S(u, t)$ be the satisfaction of a user $u$ at time $t$. Let the satisfaction be bounded on the interval $[0, 1]$ where $0$ represents complete dissatisfaction and $1$ represents complete satisfaction. Then let $\bar{S}(t)$ be the average satisfaction of all users at a time $t$. We have argued that $U(t)$ will decrease as $\bar{S}(t)$ approaches either $0$ or $1$. Additionally, it has been argued that satisfaction can increase with successful matches and decrease when there are no new profiles to swipe on. We will rephrase this second point to state that satisfaction will decrease when there are no new \textit{interesting} profiles to swipe on, as there is evidence that this is the case \citep{Tiffany2019}. Let $N(u, t)$ be the number of matches for a user $u$ at time $t$ and let $\bar{N}(t)$ be the average amount of matches made across all users at time $t$. Likewise, let $Q(u, t)$ be the number of interesting profiles available to be swiped on for a user $u$ at time $t$ and $\bar{Q}(t)$ be the average amount of interesting profiles available to be swiped on across all users. Thus, we have that $\bar{S}(t)$ will increase as $\bar{N}(t)$ increases and $\bar{S}(t)$ will decrease as $\bar{Q}(t)$ will decrease.

Note that $N(u,t)$ does not necessarily represent matches with users who possess similar preferences to the user $u$. Likewise, $Q(u,t)$ does not necessarily represent the amount of interesting profiles who possess similar preferences to the user $u$. Tinder is able to alter the output of these functions by introducing features such as Passport and Top Picks in ways such that $U(t)$ does not decrease. We will not formally define equations, but the relations between them should be noted for further study.