It does not take much to show that Tinder could meet the conditions for an optimal online dating service. Consider the preference network $P$ with the set of users $V(P)$ and the weighted, directed set of edges $E(P)$. Now let every user in Tinder's user base correspond to two $v$ for every $v \in V(P)$; one vertex for each partition. Additionally, let every $e \in E(P)$ be directed according to the possible recommendation between two users and weighted by the estimated utility determined by Tinder. Tinder allows users to specify their personal preferences, which enables the formation of communities. This lets $P$ be decomposed into the set of regular, disjoint subgraphs $H$ that can be used to satisfy the stable marriage problem. Thus, Tinder can be considered an optimal online dating service if it recommends users based on perfect matchings formed in the members of $H$. However, as per the paradox discussed previously, Tinder cannot operate both optimally and as a successful business. Therefore, there must be some form of de-optimization that allows Tinder to balance optimization with business revenue.

We can define de-optimization as any act which would prevent or discourage the formation of a stable marriage within a community. Two features of Tinder that can be considered acts of de-optimization are limitation and recycling \citep{Tiffany2019}. Limitation corresponds to limiting the amount of recommendations given to a user each day (unless the user pays a premium fee), whereas recycling refers to delivering recommendations that were previously rejected, i.e. delivering recommendations with a low estimated utility. Limitation can be considered a \textit{weak} form of de-optimization as, in an optimal online dating service, it only delays the time taken for the user to be matched within a stable marriage. Recycling can be considered weak as well as it not only serves to delay similar to limitation, but also if a user's preferences change it is possible they can now be matched within a stable marriage to the recycled recommendation.

It was noted that limitation can be overcome through a premium fee. Therefore, even though it is weak, it is still a form of de-optimization that generates revenue and helps Tinder succeed as a business. In fact, we will find that \textit{strong} forms of de-optimization are those which maximize revenue and minimize loss of user satisfaction. The strongest de-optimizations are then likely to be found in Tinder's suite of \textit{premium features} that can be accessed by paying additional fees.

Passport is one example of a premium feature. Passport is described as allowing users to receive recommendations from any location, regardless of where the user themselves are located \citep{TinderPassport}. In essence, the feature lets the user's preference for location be taken out of the calculation of the estimated utility. With less input data, the estimated utility will become less accurate. Top Picks, which recommends more desirable users \citep{TinderTopPicks}, is another example of strong de-optimization. Tinder has been known to calculate a desirability score for its users \citep{Tiffany2019}, but Top Picks will deliver recommendations that fall outside of this score, i.e. the feature removes another parameter for calculating estimated utility. 

Tinder provides additional premium features beyond these two, but these two are sufficient for showing Tinder's approach to de-optimization. Both serve to strongly affect the estimated utility of a recommendation in an adverse manner. When paired together, especially with weaker forms of de-optimzation, the effect becomes stronger. Both features also generate additional revenue for Tinder, as they require a cost to be used. However, users will receive more interesting recommendations and will want to use the service longer in order to score a match, even if the match is now less likely due to a lower estimated utility. In essence, these features discourage the formation of stable marriages while both keeping users in the user base and generating revenue from said users.

While more difficult to prove, the possibility that Tinder may not deliver ideal recommendations should also be considered. Given the set of possible recommendations 
\begin{equation*}
    I = \{i_{1}, i_{2}, \ldots , i_{k}\},
\end{equation*}
with $k$ being an integer greater than or equal to one, Tinder need only supply a subset $\hat{I} \subseteq I$ of recommendations to the user such that $\hat{I}$ is composed of recommendations with lower utility. Alternatively, if the utility estimator $\hat{R}$ factors in the utility the users pose to Tinder, the elements of $\hat{I}$ could be chosen such that if two users were matched then their utility to Tinder would be minimally affected. For example, users who match through the Passport premium feature but are unlikely to form a relationship would still generate a high estimated utility.

It should also be noted that Tinder should not discourage the formation of stable marriages too highly as doing so would risk users becoming dissatisfied with the service. A previous study has found that browsing interesting profiles and receiving matches is positively correlated with \textit{satisfaction} with Tinder \citep{Courtois2018}. Additionally, it has been found that running out of profiles to swipe on is positively correlated with \textit{dissatisfaction} \citep{Courtois2018}. The second point can be remedied with premium subscriptions that grant additional or unlimited swipe access, but the first point is the fulcrum with which Tinder must balance its algorithm. If only recommendations with low estimated utility are provided, users will become dissatisfied. However, if only recommendations with high utility are provided, users will become satisfied too quickly. All the meanwhile, Tinder must be able to maintain a steady stream of revenue.