This paper began with the paradox that Tinder cannot perform optimally without failing as a business. The stable marriage problem was applied to set theory, graph theory and network science to determine how an optimal online dating service would function. A model for the optimal online dating service was then constructed. We found that Tinder could be an optimal online dating service, but it intentionally de-optimizes itself through methods such as alternative recommendation delivery methods or, more likely, premium features.

Strong de-optimization methods such as the premium features Passport or Top Picks seem ideal for Tinder to maintain the balance between user satisfaction and business success. This paper attempts to determine possible locations for de-optimization in an optimal online dating service, but remains within the abstract for the most part. Future work may find benefits in determining how likely users are to continue using Tinder after experiencing de-optimization, especially through methods presented in premium features.

One generous assumption this paper makes is that users of Tinder are looking for long-term relationships. It is possible that users may instead be seeking friendships, short-term commitments, or some other form of interaction (e.g. just to chat). However, these alternate use cases are unlikely to impact Tinder's de-optimization methods. For example, while users who are seeking friendships are unlikely to leave the service after forming a single friendship, they are still likely to leave the service after forming enough friends as there are only so many active friendships they can sustain. As another example, a user seeking a short-term relationship may find themselves unintentionally in a long-term relationship as the result of Tinder's optimization. In either case, it is to Tinder's benefit to continue to employ de-optimization methods, whether weak or strong.

It is also possible that Tinder does not employ any de-optimization methods at all and that it is the users themselves who are responsible for the de-optimization, e.g by not messaging matches or not looking at profiles long enough to determine if they hold any appeal. This seems unlikely, and it could be argued that such methods are actually de-optimizations encouraged by Tinder, but this may be another area future work could look into.

Overall, it is not difficult to construct an overview of Tinder's algorithms, even if those algorithms remain proprietary and confidential. The question is how Tinder adjusts the parameters of these algorithms, and the answer to this question must explain how the paradox that an optimal online dating service cannot exist as a business is resolved. This paper constructs the overview and asks the question, but ultimately suggests only one possible answer. Only those who know the algorithms can answer the question truthfully, but there remains much insight to be gained by coming to our own conclusions.